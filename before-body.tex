$if(title)$
\cleardoublepage
\thispagestyle{empty}
{\centering
\hbox{}\vskip 0cm plus 1fill
{\Huge\bfseries $title$ \par}
$endif$
\vspace{12ex}
$for(by-author)$
{\Large\bfseries $by-author.name.literal$ \par}
\vspace{3ex}
{\Large Master of Public Policy \par}
\vspace{2ex}
{\Large Hertie School \par}
\vspace{20ex}
{\large Supervised by \par}
\vspace{2ex}
{\large Prof. Dr. Michaela Kreyenfeld \par}
\vspace{5ex}
{\bfseries\large $date$ \par}
\vspace{25ex}
$endfor$%
}

\newpage

%----------------------------------------------
%   Execute Summary
%----------------------------------------------

\begin{center}
{\bfseries\large Executive Summary \par}
\end{center}

\vspace*{\baselineskip}

This study explores the impact of retirement on self-reported life satisfaction among old-age workers in Germany. Utilizing data from the German Socioeconomic Panel (SOEP) linked with administrative pension records from the Deutsche Rentenversicherung (RV), the study employs a combination of fixed-effects models and regression discontinuity designs to estimate the causal effect of retirement on satisfaction with life (SWL).

The results reveal that retirement leads to a substantial increase in self-reported life satisfaction, a finding that remains robust across various methodologies and after conducting rigorous robustness tests. Moreover, early retirees show a larger increase in SWL compared to individuals who wait until reaching the statutory retirement age to retire. Additionally, the positive effect of retirement persists even beyond the initial two years following retirement. The differences by gender are not consistent across all models. To the knowledge of the author, this is the first study that analyses the impact of retirement on satisfaction with life utilizing SOEP-RV. Moreover, it provides robust evidence that retirement leads to an increase in life satisfaction, employing methodologies often omitted in the field of retirement studies, such as RDD. 

{\bfseries Key words}: retirement, individual well-being, fixed-effects, regression discontinuity designs, aging. 

\newpage